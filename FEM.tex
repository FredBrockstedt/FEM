\documentclass[ngerman]{article}
\usepackage{amsmath}
\usepackage{amsfonts}
\usepackage{amssymb}
\usepackage{amsthm}
\newtheorem*{theorem}{Theorem}
\newtheorem*{lemma}{Lemma}
\newtheorem*{prop}{Proposition}
\newtheorem*{corr}{Corrolary}
\theoremstyle{definition}
\newtheorem*{defi}{Definition}
\newtheorem*{exa}{Example}
\theoremstyle{remark}
\newtheorem*{comm}{Comment}
\usepackage[english,ngerman]{babel}
\usepackage[latin1]{inputenc}
\newcommand{\f}[2]{\frac{#1}{#2}}							%short form: \frac
\newcommand{\Cases}[1]{\begin{cases}#1\end{cases}}			%short form: \begin{cases}...
\newcommand{\Enum}[2]{\begin{enumerate}[#1]#2\end{enumerate}}	%short form: \begin{enumerate}...
\usepackage[T1]{fontenc}
\usepackage{graphicx}
\usepackage{graphics}
\newcommand{\p}{\partial}
\newcommand{\NN}{\mathbb{N}}			%creates symbol for natural numbers
\newcommand{\ZZ}{\mathbb{Z}}			%creates symbol for integral numbers
\newcommand{\II}{\mathbb{I}}				%creates symbol for irrational numbers
\newcommand{\PP}{\mathbb{P}}			%creates symbol for rational numbers
\newcommand{\QQ}{\mathbb{Q}}			%creates symbol for rational numbers
\newcommand{\RR}{\mathbb{R}}			%creates symbol for real numbers
\newcommand{\HH}{\mathbb{H}}			%creates symbol for quaternions
\newcommand{\RRn}{\RR^n}					%creates symbol for n-dim. real space
\newcommand{\CC}{\mathbb{C}}			%creates symbol for complex numbers
\newcommand{\CCn}{\mathbb{C}^n}		%creates symbol for complex numbers
\newcommand{\KK}{\mathbb{K}}			%creates symbol for a field
\newcommand{\euclid}{ \tx{euclid }}
\newcommand{\one}{\mathbb{1}}			%creates symbol for unity
\newcommand{\A}{\mathcal{A}}
\newcommand{\B}{\mathcal{B}}
\renewcommand{\L}{\mathcal{L}}
\renewcommand{\P}{\mathcal{P}}
\newcommand{\F}{\mathcal{F}}
\newcommand{\Q}{\mathcal{Q}}
\newcommand{\D}{\mathcal{D}}
\newcommand{\sig}{\sigma}
\newcommand{\case}{\begin{array}[h]{r}}
  \newcommand{\caseend}{\end{array}}
\renewcommand{\Im}{\tx{Im }}
\renewcommand{\Re}{\tx{Re }}
\newcommand{\minusone}{^{-1}}

\newcommand{\lam}{\lambda}			%creates symbol for \lambda
\newcommand{\Lam}{\Lambda}\newcommand{\GG}{\Gamma}
% creates symbol for \Lambda
\renewcommand{\aa}{\alpha}		%creates symbol for \alpha
\newcommand{\gam}{\gamma}			%creates symbol for \gamma
\newcommand{\bb}{\beta}					%creates symbol for \beta
\newcommand{\dd}{\delta}				%creates symbol for \delta
\newcommand{\ee}{\varepsilon}		%creates symbol for \varepsilon
\newcommand{\vphi}{\varphi}			%creates symbol for \varphi
\newcommand{\id}{\tx{id}}			%creates symbol for id
\newcommand{\y}{\left\langle }
  \newcommand{\yy}{\right\rangle }
\newcommand{\OO}{\Omega}
\newcommand{\ww}{\omega}
\newcommand{\DD}{\Delta}
\newcommand{\GL}{\tx{ GL }}
\newcommand{\cinf}{\tx{C}^\infty}
\newcommand{\drw}{\Rightarrow}			%creates symbol for \Rightarrow	| with reference
\newcommand{\dlw}{\Leftarrow}			%creates symbol for \Leftarrow		| to the symbol
\newcommand{\ot}{\leftarrow}
% creates symbol for \Leftarrow		| to the symbol
% creates symbol for \leftarrow		| \to looking like  -\yy 
\newcommand{\toto}{\longrightarrow}	%creates symbol for \rightsquigarrow:  ~\yy 
\newcommand{\gdw}{\Leftrightarrow}	%creates symbol for \Longleftrightarrow
\newcommand{\fa}{\ \forall}				%creates 'Space'+\forall
\newcommand{\ex}{\ \exists}				%creates 'Space'+\exists
\newcommand{\nex}{\ \nexists}			%creates 'Space'+\nexists
\newcommand{\cd}{\cdot}						%creates a short form for \cdot
\newcommand{\bs}{\backslash}			%creates a short form for \backslash
\newcommand{\bracks}[1]{\left\{#1\right\}}		%creates a brackets like {...} round the expression #1
\newcommand{\limes}[1]{\lim\limits_{#1}}%creates a \lim object and \limits beneath
\newcommand{\tri}{\nabla}
\newcommand{\End}{\tx{ End }}
\newcommand{\grad}{\tx{ grad }}
\newcommand{\supp}{\tx{ supp }}
\newcommand{\curv}{\tx{ curv }}
\newcommand{\Img}{\tx{ Img }}
\newcommand{\const}{\tx{ const }}	
\newcommand{\Span}{\tx{ span }}
\newcommand{\ric}{\tx{ Ric }}
\newcommand{\lightning}{ }
\newcommand{\rank}{\tx{ rank }}
\newcommand{\sign}{\tx{ sign }}
\newcommand{\Hom}{\tx{ Hom }}
\newcommand{\vol}{\tx{ Vol }}
\newcommand{\SU}{\tx{ SU }}
\newcommand{\SO}{\tx{ SO }}

\renewcommand{\div}{\tx{ div }}
\newcommand{\under}[2]{\underbrace{#1}{\text{#2}}}	%creates a \underbrace object defining #2 as a text
\newcommand{\stack}[2]{\stackrel{#1}{#2}}		%creates a \stackrel object defining #1 as a text
\newcommand{\tx}[1]{\text{#1}}
\newcommand{\sk}[1]{\y #1 \yy }					%creates a \text object
\newcommand{\ortho}{\bot}
\newcommand{\tbf}[1]{\textbf{#1}}
\newcommand{\tr}[1]{\tx{ tr }({#1})}
\begin{document}
\begin{titlepage}
  \begin{center}
    \ \\
    \vspace{03mm}
    {\huge Finite Element Methods\\}
    \vspace{12mm}
    {\Large by Dirk Klindworth}\\
    www.tu-berlin.de/?fem-lecture
    \vspace{12mm}\\
    {\Large  {SS 14\\ }}
    \vspace{15mm}
    {\Large written by Juri Schmelzer  \\
                         graphics by Fred Brockstedt}
  \end{center}
\end{titlepage}
\newpage
\tableofcontents
\newpage
\section{Theoretically Background}
\begin{defi} a differential equation is an equation satisfied by a function $u$, which involves besides $u$ its derivates
  \begin{enumerate}
  \item $u$ depends on only one variable (ordinary differential equatations)
  \item $u$ depends on more than one variable ($\p_i u:= \f{\p u (z)}{\p z_i}$ partial differential equations (PDE))
  \end{enumerate}
\end{defi}
\subsection{Classification of PDE}
elliptic, hyperbolic, parabolic

\begin{enumerate}
\item elliptic: an incident at $z$ influences all points in the neighbourhood.
\item parabolic: there is one direction from $z$ where the influence is only for larger values
\item there are areas of influence
\end{enumerate}
\begin{defi} The principle part of the secound order PDE
  $$-\sum_{i,j=1}^n \p_i a_{i,j}(z)\p_j u(z) + \sum_{i=1}^n b(z)\p_iu(z) + c(z) u(z) = f(z)$$
  is $$\sum_{i,j=1}^n \p_i a_{i,j}(z) \p_j u_j(z)$$
\end{defi}
\begin{defi}
  \begin{enumerate}
  \item the PDE is elliptic at z if the eigenvalues of A are all non-zero and of the same sign
  \item parabolic if one eigenvalue is 0 and the rest all positive or all negativ
  \item hyperbolic if n-1 eigenvalues have the same sign and on has the other sign
  \end{enumerate}
\end{defi}

\begin{exa} Stationary heat equation\\
  In absence of work the conservation of energy correspondens to the coversation of temperature
  $$f{\p u}{\p t}(x,t) + \tri j (x,t)= f(x,t)$$
  where
  \begin{enumerate}
  \item u is the temperature with unit [u]=1K
  \item j is the heat flux with unit [j]=$1\f{W}{m^2}$
  \item f is a heat source with unit [f]=$1\f{W}{m^3}$
  \end{enumerate}
  As constituive equation (material property) we have Fouriers law 
  $$j(x,t)=-A(x)\tri u(x,t)$$
  We call the material 
  \begin{enumerate}
  \item homogenous if $A(x)=A$, inhomogenous otherwise
  \item isotropic if $A(x)= \aa(z)1,$ anisotropic otherwise
  \end{enumerate}
\end{exa}
For a unique solution u of constituve equation we need initial conditions
$$u(x,0)= u_0 (x)$$
and boundary conditions on boundary $\p \OO$ of $\OO$
\begin{enumerate}
\item $u(x,t)= g(x,t)$ for $x \in \p \OO, t>0$ (Dirichlet b.c.)
\item $j(x,t) n(x) = h(x,t)$ for $x \in \p\OO, t>0$ (Neumann b.c.)
\item $j(x,t) n(x) + \bb(x) u(x,t) = h(x,t)$ (Robin b.c.)
\end{enumerate}
Static emit of constituive equatation.\\
we sppose that the temperature does not change in time i.e. $\f{\p u}{\p t}=0$
$$\drw \tri j(x) = f,\ j(x)=-A \tri u \drw -\tri A(x)\tri u(x) = f(x) + b.c.$$


other examples\\
The same system arises in 
\begin{enumerate}
\item electro static, where u is the electric potential, j is the displacement field (usually denoted by D), f is the charge density (usually denoted by $\rho$)
\item stationary electric currents, where u is the electric potential and j is the electric current, f=0
\end{enumerate}
Well posednesi\\
\begin{defi} A problem is said to be well posed if 
  \begin{enumerate}
  \item it has a unique solution $u$, and
  \item the solution depends continously on the given data f i.e. $||u||\leq C ||f||$
  \end{enumerate}
\end{defi}
Recall transient/stationary heat equation
$$\underbrace{\f{\p u}{\p t}}_{=0 \tx{in case of stationary h.eq}}+ \nabla j = f$$
$$j= -A(x)\nabla u$$
with b.c's
\begin{enumerate}
\item $u(x) = g(x)$ (Dirichlet)
\item $j(x) n(x) = h(x) $ (Neumann)
\item $j(x)n(x) + \bb (x) u(x) = h(x) $ (Robin)
\end{enumerate}
\subsection{Elliptic boundary value problem}
\begin{comm}abstract notation:\\
  \begin{align*}
    -\nabla a \nabla u + c u &= f& \tx{in } \OO\subset\RR^d,d=1,2,3\\
    u&=g & \tx{on }\GG_D \subset \p \OO\\
    a\nabla u n&=h &\tx{on } \GG_N \subset \p \OO \\
    a \nabla u n + \bb u &= h & \tx{on }\GG_{R}\subset \p \OO
  \end{align*}
  $Re(c)\geq 0$, $0<\aa_0\leq Re(a) \leq a \leq \infty$\end{comm}
\begin{defi}classical Solution:\\
  If $u \in C^2(\OO)\cap C(\bar\OO)$ satisfies the PDE as well as the b.c's in a pointous sense, then u is calles classical solution of the BVP
\end{defi}
\begin{comm}
  \begin{enumerate}
  \item existence of classical solution is a requiremten of some numerical scheme, e.g. the finite difference methode (FDM)
  \item this requirement will be weakend when introducing the variational formulation.
  \end{enumerate}
\end{comm}
\begin{corr}
  Small perturbation in the data of a linear, well posed problem lead to small perturbation of the solution
\end{corr}
Proof: $\tilde f = f + \delta f , \delta f <<1$, then $atilde u$ solves the prolem with data $\tilde f$ and $\delta u = \tilde u - u$ solves the problem with the data of $\drw ||\delta u || \leq C||\delta f||$

\begin{exa} how to proof well-posed (Serial 1 1.b)\\
  find $u(x,1) $ with initial data $sin(n\pi x) ,\ n \in \NN$
  \begin{align*}
    \tilde u (x,0) &= sin(\tilde n \pi x)& \tx{subst. } s=\tilde n \pi x\\
    ||\tilde u (x,0)||^2_{L^2(-1,1)} &= \int^1_{-1}sin^2(\tilde n \pi x) dx &=... = 1\\
    \tilde u (x,t=1) &= e^{-\tilde n^2\pi^2}sin(\tilde n \pi x)\\
    ||\tilde u (x,1)||_{L^2}&= \sqrt{e^{-\tilde n ^2 \pi^2}}||\tilde u(x,0)||_{L^2} 
    &\leq \underbrace{C}_{=1} ||\tilde u (x,0)||_{L^2} &\quad
  \end{align*}
\end{exa}
\subsection{Numerical solutions of elliptic BVP}
\begin{comm}
  Closed form solution versus nummerical solution
  \begin{enumerate}
  \item only most elementary PDE have closed form solutions
  \item but even if they exist, there value can be questioned
  \end{enumerate}
\end{comm}
\begin{exa}
  Consider Poisson Problem equation\\
  \begin{align*}
    -\tri u &= 2 & \tx{in }\OO\in (-1,1)^2\\
    u&=0 & \tx{on }\p\OO
  \end{align*}
  The closed form solution looks like this:

  $$u(x,y) = 1 - y^2 -\f{32}{\pi^3} \sum_{n=1}^\infty \f{(-1)^k cosh(\f{2k+1}{2}\pi x) cos(\f{2k+1}{2}\pi y)}{(2k+1)^3cosh(\f{2k+1}{2}\pi)}$$
  Problems:
  \begin{enumerate}
  \item infinite summ need to be truncated: very often the number of summands is prohibitivly large to sufficent accurary
  \item for large k $$\f{cosh(\f{2k+1}{2}\pi x)}{cosh(\f{2k+1}{2}\pi)}$$
    the nummerical calculations is far from trivial.
  \item A modification of the b.c's may result in a totally diffrent schemes.
    
  \end{enumerate}
\end{exa}
\subsubsection{Finite Diffrence Method (FDM)}
If a classical solution of the BVP exists (which implies that $u\in C^2(\OO)$), we can replace the derivatives by different quotients.\\

Let d=1 and $\OO\subset \RR$ be an interved, h>0, Then, if $u \in C^{n+1}(\OO)$ then we can employ the  Taylor theorem, which gives:
$$u(x\pm h) = u(x)\pm hu'(x) + \f{h^2}{2}u''(x) + ...+\f{\pm h^n}{n!}u^{(n)}(x)+ R_n(u,x,h)$$
with the remainder 
$$R_n(u,x,h)=\f{1}{n!}\int^{x\pm h}_x (x-t)^n u ^{n+1}(t)dt= \f{\pm h^{n+1}}{(n+1)!} u^{(n+1)(\eta)}$$
for some $\eta \in [x,x\pm h]$\\
\begin{enumerate}
\item $\drw$ first order forward diffrence quotient\\
  $$\f{u(x+h) -u(x)}{h}= u'(x) + \O(h)$$
\item $\drw$ first order backward diffrence quotient\\
  $$\f{u(x) -u(x-h)}{h}= u'(x) + \O(h)$$
\item $\drw$ secound order centered diffrence quotient\\
  $$\f{u(x+h) -u(x-h)}{2h}= u'(x) + \O(h^2)$$
\item $\drw$ standard d.q.\\
  $$\f{u(x+h) -2u(x) +u(x-h)}{h^2}= u''(x) + \O(h^2)$$
\end{enumerate}
advantages of FDM
\begin{enumerate}
\item easy to implement
\item system matrix is sparse
\item convergent solution in $l_\infty$-norm
\end{enumerate}
disadvantages of FDM
\begin{enumerate}
\item required of classical solution
\item a(x) needs to be continously diff.
\item limitation to simple domains
\item pointwise approximation
\end{enumerate}
\subsubsection{Finite Element Method}
\begin{enumerate}
\item weaker requirement on the regularity of the solution/ material function
\item possibility for more complicated domains
\end{enumerate}
\begin{comm}
  Key indrigents
  \begin{enumerate}
  \item transfomration of BVP from strong formation to a so-called variational (weak) formation
  \item Discretization of the computational domain $\drw$ irregular meshes instead of grid
    \begin{itemize}
    \item based on mesh we define basis functions of discrete supspace of the space where we loof for the weak solution
    \item basic solutions have a local support
    \end{itemize}
  \end{enumerate}
\end{comm}
\section{Variational Formation}
\subsection{computational domain}
\begin{defi}
  \begin{itemize}
  \item domain $\OO$: bounded, connected, open subset of $\RR^d$, d=1,2,3
  \item boundary $\p\OO:=\bar\OO \backslash \OO$ at least $C^0-$continous and closed.
  \end{itemize}
  Furthermore, the domain needs to be a Lipschitz-domain
  \begin{itemize}
  \item boundary is of finite length (excluding fractal boundaries)
  \item boundary slightly smoother than continous (excluding slit/cusp domains)
  \end{itemize}
\end{defi}
\begin{comm}
  In practise only domains can be described by CAD software are relevant. This set of domains is usually equivalent to the following set of domains\end{comm}

\begin{defi}
  Let d=2. A connected domain $\OO$ is called \emph{curvilinear Lipschitz-polygon} if 
  \begin{itemize}
  \item $\OO$ is Lipschitz
  \item there is a finite number of open subsetd $\GG_K\subset \p\OO, k=1,...,p\in \NN$, such that 
    $$\p\OO=\bar\GG_1\cup....\cup \bar\GG_p,\ \GG_k\cap \GG_l=\emptyset \ \forall k\neq l$$
    and for each $k\in\left\{1,...,p\right\}$ there exists a $C^1-$diffeomorphism (invertible mapping of smooth manifolds) $\Phi_k : [0,1]\to \GG_k$
  \end{itemize}
\end{defi}
\begin{defi}
  A subset $\OO\subset \RR^d$ is called \emph{computational domain} if it is bounded and its boundary is $C^1-$continous, or if it is 
  \begin{itemize}
  \item bounded connected interval(d=1)
  \item a curvilinear Lipschitz-polygon(d=2)
  \item a curved Lipschitz-polyhedron(d=3)
  \end{itemize}
\end{defi}
\subsection{Linear differential operator}
\begin{defi}
  Let $\aa \in \NN_0 ^b$ be a multi-index and $|\aa| = \aa_1+...aa_d$. Then we call 
  \begin{itemize}
  \item $$\p^\aa:=\f{\p^{\aa_1}}{\p x_1^{\aa_1}}...\f{\p^{\aa_d}}{\p x_d^{\aa_d}}$$
    the \emph{partied deriviate of order} $|\aa|$\\
  \item the \emph{gradient} of a function f 
    $$\nabla f (x) = (\f{\p f}{\p x_1}(x),...,\f{\p f}{\p x_p}(x))^{(T)}$$
  \item \emph{divergenz} of a vector field $ f=(f_1,...,f_d)^{(T)}$
    $$\nabla \cdot f(x) = \sum_{i=1}^d \f{\p f_i}{\p x_i}(x)$$
  \item \emph{Laplacian}
    $$\Delta :=\tri \cdot \tri$$
  \end{itemize}
\end{defi}
\subsection{Integration by parts}
\begin{theorem} Gau\ss{}-Theorem\\
  If $f \in(C^1(\OO))^d\cap (C^0(\bar\OO))^d$ then 
  $$\int_\OO \tri \cdot f d\ x = \int_\GG f \cdot n d\ s$$
  By the product rule $\tri \cdot (u f )= u \tri \cdot f + \tri u \cdot f$ we deduce the Grenne formula
  $$\int_\OO f \cdot \tri u + \tri \cdot f u d\ x = \int_\GG f \cdot n u d\ s$$
  for all $f \in(C^1(\OO))^d\cap (C^0(\bar\OO))^d$ and $u \in(C^1(\OO))^d\cap (C^0(\bar\OO))^d$
\end{theorem}
\begin{exa}
  \begin{itemize}
  \item $f=f e_k$ where $e_k$ is the k-th unit vector
    $$\drw \int_\OO f \f{\p u}{\p c^k}+ \f{\p f}{\p x_k} u d\ x = \int_\GG f u(n)_k d\ s$$
  \item $f = \nabla v, v \in C^2(\OO)\cap C^1(\bar\OO)$
    $$\drw \int_\OO \nabla u \cdot \nabla v + \Delta v u d\ x = \int_\GG \nabla v n d\ s$$
    
  \end{itemize}
\end{exa}
\subsection{Disribuitional derivations}
\begin{defi} Test function space\\
  For a non-empty open set $\OO \subset \RR^d$ we denote 
  $$\cinf_0 (\OO) := \left\{v \in \cinf(\OO) : supp\ v := \left\{x \in \OO: v(x)\neq 0 \right\} \subset \OO\right\}$$
  the space of test functions
\end{defi}
\begin{lemma}
  Two integral functions f and g defined in the bounded set $\OO$ are almost everywhere equal 
  $$\gdw \int_\OO f v d\ x = \int g v d\ x$$
  for all test functions $v \in \cinf_0(\OO	)$
\end{lemma}
\begin{defi}
  Let $u \in L^2(\OO)$ and $\aa \in \NN^d _0$. A function $w \in L^2(\OO)$ is calles \emph{weak derivate} or \emph{distributioned derivate} $\p^\aa u$ (of order $|\aa|$) of u, if 
  $$\int w v d\ x = (-1)^{|\aa|} \int _\OO u \p^\aa v d\ x$$
  for all $v \in \cinf_0(\OO)$

\end{defi}

\begin{itemize}
\item weak gradient $w \in (L^2(\OO))^d$ of $u \in L^2(\OO)$
  $$\int w\cdot v d\ x = -\int u \nabla \cdot v d\ x$$
  for all $v \in (\cinf_0 (\OO))^d$
\item weak divergence $w\in L^2(\OO)$ of $u \in (L^2(\OO))^d$
  $$\int w\cdot v d\ x = -\int u \cdot\nabla  v d\ x$$
  for all $v \in \cinf_0(\OO)$
\end{itemize}
\begin{theorem}
  If $u \in C^m(\bar\OO)$ then al weak derivates of order $\leq 0$ agree in $L^2(\OO)$ with the corresponding classical derivates.
\end{theorem}
\begin{comm}Extension of integration by parts\\

  For Lipschitz domaiuns the Gau\ss{}-theorem (and hence also (7G7)) can be extended to integrable functions with weak derivates
\end{comm}
\begin{comm} Weak derivates in subdomains\\
  The weak derivates in $\OO$ and in a subdomain $\OO_1 \subset \OO$ coincede in $\OO_1$ since the space of test functions satisfy $\cinf_0(\OO) \subset \cinf_0(\OO_1)$
\end{comm}
\begin{comm} Continuity over interfaces of subdomains\\
  For the partition $\bar\OO = \bar\OO_1 \cup \bar\OO_2, \OO_1\cap\OO_2=\emptyset,$ where both subdomains are supposed to have a Lipschitz boundary, a function u, which has a weak gradient in $\OO_1$ and $\OO_2$has a weak gradient in $\OO$ if and only if $[u]_2 = 0$ $$\sum:= \p \OO_1 \cap\p \OO_2$$
\end{comm}
\begin{defi} extra remarks to computational domain\\
  An open, connected set $\OO \subset \RR^d $ is called \emph{Lipschitz-domain}, if its boundary $\p\OO:=\bar\OO\backslash \OO$ is \emph{Lipschitz boundary} which is equivalent to 
  \begin{itemize}
  \item $\exists$ finite vovering of $\p\OO$ of open subsets of $\RR^d$, and 
  \item $\p \OO$ i locally (for each open rectangle) the graph of a Lipschitz continous function
  \end{itemize}
\end{defi}
\begin{defi}
  curvi linear lipschitz polygon\\
  d=2 \ A Lipschitz domain $\OO\subset \RR^d$ is called a c.L.p. if $\exists $ finite number of open subsets $\GG_k \subset \p \OO$, $k=1,...,P, P\in \NN$ s.t. $\forall k \exists C^1$-diffeomorphism $\Phi_k : [0,1] \to \GG_k$

\end{defi}
\subsection{Variational formulations of elliptic BVPs}
\begin{comm}
  Recall BVP 1.4
  \begin{align}
    -\tri \cdot a \tri \cdot u + su &= f& in\ \OO\\
    u&=g &on\ \GG_D
    a \tri u \cdot n &= h& on\ \GG_N\\
    a\tri u \cdot n + \bb u &= h& on\ \GG_R
  \end{align}
  and Eq(2.4), weak divergence $w \in L^2(\OO)$ of $u \in (L^2(\OO))^d$
  $$\int_\OO w v d\ x = - \int_\OO u \cdot \tri v d\ x \ \forall v \in \cinf_0(\OO)$$
\end{comm}
\subsubsection{Pure Dirichlet b.c.'s}
\begin{comm}
  $$\int_\OO (-\tri \cdot a \tri u + cu)v d\ x = \int_\OO fv d\ x \ \forall v\in\cinf_0(\OO)$$
  $$\overbrace{\drw}^{2.4} \int_\OO a \tri u \cdot \tri v + cuv d\ x = \int_\OO fv d\ x \ \forall v\in\cinf_0(\OO)$$
  $u=g$ becomes an \emph{essential b.c.}, i.e. it is incorporated into the space of test and trial functions
\end{comm}
\subsubsection{Pure Neumann b.c.'s}
$$\int_\OO (-\tri \cdot a \tri u + c u) v d\ x = \int_\OO f v d\ x \ \forall v \in \cinf(\bar\OO)$$
FGF
$$\int_\OO f \cdot \tri u + \tri \cdot f u d\ x = \int_\GG f\cdot n u d\ S(x)$$
$$f= a \tri u \drw \int_\OO a\tri u \cdot \tri v d\ x - \int_\GG a \tri u \cdot n v d\ S(x)=-\int_\OO \tri \cdot a \tri u v d\ x$$
$$\drw \int_\OO a \tri u \cdot \tri v + c u v d\ x - \int_\GG \underbrace{a\tri u \cdot n}_{=h} v d\ S(x) = \int_\OO f v d\ x \ \forall v\in\cinf(\bar\OO)$$
Boundary condition which are present in the variational formulation are called natural b.c.
\subsubsection{Robin b.c.'s}
$$\int_{\OO}a \tri u \cdot \tri v + c u v d\ x - \int_\GG \underbrace{a \tri u}_{h-\bb u} \cdot n v d\ S(x) = \int_\OO f v d\ x$$
$$\int_\OO a \tri u \cdot \tri v + c u v d\ x + \int_\GG \bb u v d\ S(x) = \int_\OO f v d\ x + \int_\GG g v d\ S(x) \forall v \in \cinf (\bar\OO)$$







? order of the next part?
\begin{exa} 
  $\OO= (0,1).$ Give an example of a function $u \in C^1(\OO)$ that does not posses a weak gradient bounded in $L^2(\OO)$\\
  Recall Thm 2.7.\\
  If $u \in C^m(\bar\OO)$, then all weak derivates of order $\leq m$ agree in $L^2(\OO)$ with the classical deriate\\
  For example:
  \begin{align*}
    f(x) &= ln\ x\\
    f'(x)&=\f{1}{x}\\
    ||f'||_{L^2}^2&= \int_0 ^1 (f')^2 d\ x\\
    -f(x) = \sqrt{x}
    \vdots
  \end{align*}
\end{exa}
%% missing part added by Fred
\subsection{Linear and bilinear forms}
Let $V,W$ be real/complex vector spaces
\begin{itemize}
\item \index{Linear operator}\emph{Linear operator} is a mapping
  \begin{eqnarray*}
    T:V & \to & W\\
    T\left(\lambda v+\mu w\right) & = & \lambda T\left(v\right)+\mu T\left(w\right)
  \end{eqnarray*}
  $\forall v,w\in V$ and $\lambda,\mu\in\mathbb{R}$ or $\mathbb{C}$.
\item \index{Linear form}\emph{Linear form} is a linear operator $l:V\to\mathbb{R}$
  or $\mathbb{C}$.
\item \index{Bilinear form}\emph{Bilinear form} is a mapping
  \[
  b:V\times V\to\mathbb{R}\mbox{ or }\mathbb{C}
  \]
  where $v\mapsto b\left(v,w\right)$ $\forall w\in V$ and $w\mapsto b\left(v,w\right)\forall v\in V$
  are linear forms
  \[
  \underset{=b\left(u,v\right)}{\underbrace{\int_{\Omega}a\nabla u\cdot\nabla v+cuv\ dx}}=\underset{=l\left(x\right)}{\underbrace{\int_{\Omega}fv\ dx}}
  \]

\item \index{Antilinear operator}\emph{Antilinear operator} is a mapping
  \begin{eqnarray*}
    T:V & \to & W\\
    T\left(\lambda v+\mu w\right) & = & \bar{\lambda}T\left(v\right)+\bar{\mu}T\left(w\right)
  \end{eqnarray*}
  $\forall v,w\in V$, $\forall\mu,\lambda\in\mathbb{C}$.
\item \index{Antilinear form}\emph{Antilinear form} is a antilinear operator
  $l:V\to\mathbb{C}$.
\item \index{sesquilinear form}\emph{sesquilinear form} is a mapping
  \begin{eqnarray*}
    b:V\times V & \to & \mathbb{C}
  \end{eqnarray*}
  such that $v\mapsto b\left(v,w\right)$ is a linear form $\forall w\in V$
  and $w\mapsto b\left(v,w\right)$ is a antilinear form
\end{itemize}

\paragraph{Properties of bi-/sesquilinear forms}
\begin{itemize}
\item symmetric $b\left(u,v\right)=b\left(v,u\right)$
\item positive definite $\left|b\left(v,v\right)\right|=0\iff v\neq0$
\item V-continuous (or V-bounded) $\left|b\left(u,v\right)\right|\le\underset{\mbox{continuity constant}}{\underbrace{\left|\left|b\right|\right|}\left|\left|u\right|\right|_{V}\left|\left|v\right|\right|_{V}}$
  $\forall u,v\in V$
\item V-elliptive (also coersive) $\left|b\left(v,v\right)\right|\ge\underset{\mbox{ellipticity constant}}{\underbrace{\gamma}\left|\left|v\right|\right|_{V}^{2}}$
  $\forall v\in V$
\end{itemize}
A symmetric positive definite bi-/sesquilinear form is an \emph{inner
  product\index{inner product}} that induces a norm
\[
\left|\left|v\right|\right|_{b}:=\left|b\left(v,v\right)\right|
\]
In the context of elliptic PDEs, if the bi-/sesquilinear form $b$
in (LVP) is symmetric, positive definite, it is called \emph{energy
  norm}\index{energy norm} $\left|\left|\cdot\right|\right|_{e}$.

%% missing part end

  \subsection{Sobolev spaces}
  \begin{defi}
    A normed vector space V is complet, if every Cauchy sequence $\left\{ v_k \right\}_k \subset V$ has a limit in V. A normed, complete vector space is called Banach space\end{defi}
  \begin{defi}
    A Hilbert space is a Banachspace whose norm is induced by a inner produkt.\end{defi}
  Recall variational formulation of BVP with pase, homogenous Neumann b.c.
  $$\int_\OO a \tri u \cdot \tri v + cuv d\ x = \int_\OO fv d\ x$$
  for all test functions v$\in \cinf _0(\OO)$\\
  \emph{Ideal space} of u and v in 2.9. is a Hilbert space whose inner product coincedes with the bilinear form, i.e. it is equipped with the energy norm
  $$||v||_e ^2 = \int |a||\tri v |^2 +|c||v|^2 d\ x $$
  and the space is defined as 
  $$ H:= \left\{v : \OO \to \RR | \tx{weak gradient }\tri v \tx{ exists and } ||v||_e <\infty \right\}$$
  The same spplies also to the test function, that are now more functions than in $\cinf_0$(equatation due to Meyser-Serrin Theorem, see Thm.2.15)
  \begin{defi}
    The sobolev space $H^1(\OO)$ is the space of all square integrable functione with square integrable weak gradients
    $$H^1(\OO) := \left\{v\in L^2(\OO) : \tx{weak gradient }\tri v \tx{ exists and } ||\tri v||_{L^2} <\infty \right\}$$
    with norm $$||v||^2 _{H^1(\OO)} = ||v||^2 _{L^2(\OO)} + D^2 ||\tri v||_{L^2(\OO)}$$
    where $D= diam (\OO)$ to meet unit ....


    The $H^1$-norm and the energy norm are equivaled if the latter is base on a $H^1-$elliptic and $H^1-$continous bilinear form, i.e. $\exists C_1,C_2>0$ such that:
    $$C_1 ||v||_e \leq ||v||_{H^1 }\leq C_2||v||_e \ \forall v \in H^1(\OO)$$
  \end{defi}
  \begin{defi} For $m \in \NN_0$ and $\OO\subset \RR^d$ we define the sobolev of order m as ...
  \end{defi}

  \begin{theorem} 2.15. Meyser-Serrin theorem\\
    The space $C^\infty(\bar\OO)$is a dense subspace of $H^m(\OO)$ for all $m \in \NN_0$
  \end{theorem}
  \begin{defi}
    $H^1 _0 (\OO)$ is defined as the completation of $\cinf_0(\OO)$ with respect to the $H^1-$norm
  \end{defi}
  \begin{exa}
    Consider
    \begin{align*}
      -\tri \cdot a \tri u + cu & = f & \in \OO \\
      u                         & =0  & on\ \p\OO
    \end{align*}
    solutionway:
    \begin{align*}
      (a) & \int_\OO a \tri u \cdot \tri v + cuv d\ x & = \int_\OO fv d\ x + \int_{\p\OO} \tri u \cdot n v d\ s(x) & \forall v \in \cinf_0(\OO)                 \\
      (b) & ||v||_e ^2                                & := \int_\OO a \tri v \cdot \tri v + cvv d\ x                                                            \\
          & ||v||_e ^2                                & \geq 0                                                     & \forall v \in V=H_0 ^1 \drw a,c\geq \gg >0 \\
      (d) & \tx{Is }b \ H^1 _0-\tx{elliptic}?                                                                                                                   \\
          & \exists \gg>0:|b(v,v)|                    & \geq \gg_e||v||_{H^1}^2                                    & \forall v \in H_0 ^1(\OO)                  \\
          & |b(v,v)|                                  & \geq\gg\int_\OO (\tri v)^2 + v^2 d\ x                                                                   \\
    \end{align*}
  \end{exa}
  \subsection{Theory of variational formulations}
  \subsubsection{Elliptics and the Lemma of Lax-Milgram}
  \begin{defi} The dual V' of a normal vector space V is the normed vector space of continous linear forms on V. The dual space is equipped with the operator norm 
    $$||f||_{V'} := sup\ _{v \in V\backslash \left\{0\right\}} \f{f(v)}{||v||_V}$$
  \end{defi}
  \begin{theorem}2.21 Lemma of Lax-Milgram\\
    Let V be a reflexive Banach space. Let bilinear form $b:V \times V \to \RR$ or sequilinear form $b:V\times V \to \CC$ be V-elliptic. Then, the variational problem (LVP), i.e. $b(u,v) = l(v)$ has for any $f \in V'$ a unique solution $u \in V$ with 
    $$||u||_V \leq \f{1}{\gg_e}||f||_{V'}$$
  \end{theorem}
  \subsubsection{ The inf-sup conditions }
  \begin{theorem}
    The following staetements are equivalent
    \begin{enumerate}
    \item For all $f \in V'$ the linear variational Problem (LVP) has a unique solution $u \in U$ that satisfies 
      $$||u||_U \leq \f{1}{\gg}||f||_{V'}$$
      with $\gg>0$ independent of f\\
    \item The bilinear form b satisfies the \emph{inf-sup-conditions}
      $$\exists \gg >0: inf\ _{w \in U \backslash\left\{0\right\} } sup\ _{v \in V \backslash \left\{0\right\}} \f{|b(w,v)|}{||v||_v||w||_U} \geq \gg \ (IS1)$$
      $$\forall v \in V \backslash\left\{0\right\} : ... sup\ _{w \in U \backslash\left\{0\right\}} b(w,v)>0 \ (IS2)$$
    \end{enumerate}
  \end{theorem}
  
  \end{document}